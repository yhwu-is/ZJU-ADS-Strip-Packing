\documentclass[12pt]{article}

\usepackage[a4paper, left=3.17cm, right=3.17cm, top=2.54cm, bottom=2.54cm]{geometry}
\usepackage[T1]{fontenc}
\usepackage{mathptmx}
\usepackage{amsmath}
\usepackage{amsfonts}
\usepackage{chemformula}
\usepackage{cite}
\usepackage[colorlinks, linkcolor=black, anchorcolor=black, citecolor=black]{hyperref}
\usepackage{graphicx}
\usepackage{fancyhdr}
\usepackage{enumitem}
\usepackage{listings}
\usepackage{setspace}
\usepackage[a4paper]{geometry}
\usepackage[ruled]{algorithm2e}

\geometry{left=1.8cm, right=1.8cm, top=3cm, bottom=2.5cm}

\setlength{\parskip}{0.5em}
\title{Strip Packing \\ \bigskip \large Project6\ \ Report\ \ by\ \ Group1}
\author{Guo Jiahao \\ Wu Yihang \\ Sun Xinjie}

\lstset{
    columns=fixed,       
    numbers=left,                                        % 在左侧显示行号
    numberstyle=\tiny\color{gray},                       % 设定行号格式
    frame=none,                                          % 不显示背景边框
    backgroundcolor=\color[RGB]{245,245,244},            % 设定背景颜色
    keywordstyle=\color[RGB]{40,40,255},                 % 设定关键字颜色
    numberstyle=\tiny,keywordstyle=\color{blue!70},
    commentstyle=\color{red!50!green!50!blue!50},frame=shadowbox,
    rulesepcolor=\color{red!20!green!20!blue!20},basicstyle=\ttfamily,
    stringstyle=\rmfamily\slshape\color[RGB]{128,0,0},   % 设置字符串格式
    showstringspaces=false,                              % 不显示字符串中的空格
    language=C++,                                        % 设置语言
    }

\pagestyle{fancy}
\fancyhf{}
\cfoot{\thepage}
\fancyhead[C]{Advanced Data Structure and Algorithm Analysis\ \ \ \ Project6  Report\ \ \ \ Group1}
\begin{document}

\begin{titlepage}
	\newcommand{\HRule}{\rule{\linewidth}{0.5mm}}
	\begin{figure}
        \flushleft
        \includegraphics[scale=0.4]{0.png}
    \end{figure}
    \center 
	\quad\\[1.5cm]
	\textsl{\Large Zhejiang University }\\[0.5cm] 
	\textsl{\large College of Computer Science and Technology}\\[0.5cm] 
	\makeatletter
	\HRule \\[0.4cm]
	{ \huge \bfseries \@title}\\[0.4cm] 
	\HRule \\[1.5cm]
	\begin{minipage}{0.4\textwidth}
		\begin{flushleft} \Large
			\emph{Author:}\\
			\@author 
		\end{flushleft}
	\end{minipage}
	~
	\begin{minipage}{0.4\textwidth}
		\begin{flushright} \Large
			\emph{Supervisor:} \\
			\textup{Mao Yuchen}
		\end{flushright}
	\end{minipage}\\[3cm]
	\makeatother
	\begin{flushleft}
        \Large
        \textbf{Abstract:}

    \end{flushleft}
    {\large An Assignment submitted for the ZJU:}\\[0.5cm]
	{\large {21120491\ \ Advanced Data Structure and Algorithm Analysis}}\\[0.5cm]
	{\large \today}\\[2cm] 
	\vfill 
\end{titlepage}
    
    \section{Introduction of the Project}
    In this project, we need to solve the strip packing problem.
    The strip packing problem is a 2-dimensional geometric minimization problem.
    Given a set of axis-aligned rectangles and a strip of bounded width and infinite height,
    determine an overlapping-free packing of the rectangles into the strip minimizing its height.
    We may assume that the width of any rectangle is no more than the width of the bin,
    and we are not allowed to rotate the rectangles, and that the rectangles should not overlap.

    Note that we need to implement at least two polynomial-time approximation algorithms for this problem.
    We must generate test cases of different sizes with different distributions of widths and heights,
    compare and analysis the solution quality and the running time of the two algorithms on these test cases. 

    \section{Introduction of the Algorithms}
    \subsection{Next Fit}
    We first discuss the simplest method to solve this problem which called next fit algorithm.
    In this algorithm, We place the items one by one.
    When the bottom edge of the next item cannot be placed at the same height as the previous item,
    we cover the "cover" of the previous layer, that is,
    place the bottom edge of the next item on the height of the top of the highest item on the current layer.
    Pseudo code is shown below:
    \begin{algorithm}
        \caption{Next Fit}
        \LinesNumbered
        \KwIn{width \emph{w} and height \emph{h} of input items}
        let \emph{y = 0}\;
        let \emph{x = 0}\;
        let \emph{h = 0}\;
        \For{Rectangle R = (w, h)\textup{in the sequence}}{
            Try to fit the rectangle onto the current open shelf\;
            \If{\textup{It does not fit}}{
                close the current shelf and open a new one\;
            }
        }
    \end{algorithm}
    
    \begin{figure}[h]
        \centering
        \includegraphics[scale=0.7]{heap.png}
        \caption{An example of a Fibonacci heap}
    \end{figure}
    \subsection{Implementation of Operations}
    \subsubsection{Insertion}
    Insertion for Fibonacci heap is a lazy operation. The following procedure inserts node \emph{x} into Fibonacci heap \emph{H}:
    \begin{algorithm}
        \caption{Insertion for Fibonacci heap}
        \LinesNumbered
        \KwIn{Fibonacci heap \emph{H}, node \emph{x}}
        \emph{x.degree} = 0\;
        \emph{x.p} = \emph{x.child} = NULL\;
        \emph{x.mark} = false\;
        \eIf{H.min == \textup{NULL}}{
            create a root list for \emph{H} containing just \emph{x}\;
            \emph{H.min} = \emph{x}\;
        }{
            insert \emph{x} into \emph{H}'s root list\;
            \If{x.key < H.min.key}{
                \emph{H.min = x}\;
            }
        }
        \emph{H.n} = \emph{H.n} + 1
    \end{algorithm}

    First we initialize some attributes of node \emph{x}. Next, we test if Fibonacci heap \emph{H} is empty. 
    If it is, we make \emph{x} be the only node in \emph{H}’s root list and set \emph{H.min} to point to \emph{x}. 
    Otherwise, we insert \emph{x} into \emph{H}’s root list and update \emph{H.min} if necessary. Finally, we increment \emph{H.n}.

    \subsubsection{Make heap}
    For the making heap operation of Fibonacci heap, we just need 
    to insert n keys continuously.

    \subsubsection{FindMin}
    Because we maintain a pointer to the minimum node of the heap, 
    so we only need to return \emph{H.min.data}.
    \subsubsection{DeleteMin}
    DeleteMin is much more complicated than insertion. It works by first making a root
    out of each of the minimum node’s children and removing the minimum node from
    the root list. It then consolidate the root list by linking roots of equal degree until
    at most one root remains of each degree. The pseudocode of Algorithm 2 extracts the minimum node.
    
    The procedure consolidate uses an auxiliary array \emph{A}, whose size can be limited by the bounding the maximum degree
    $$D(n) \le \lfloor\log_{\phi}n\rfloor,\quad \phi = (1 + \sqrt{5}) / 2$$
    , to keep track of roots according to their degrees. If \emph{A}[\emph{i}] = \emph{y}, then \emph{y} is currently a root
    with \emph{y.degree = i}.

    In detail, let's see the pseudocode of Algorithm 3, the consolidate procedure works as follows. 
    We first allocate array \emph{A}, then we link roots together, \emph{w} may be linked
    to some other node and no longer be a root. Nevertheless, \emph{w} is always in a tree
    rooted at some node \emph{x}, which may or may not be \emph{w} itself. Because we want at
    most one root with each degree, we look in the array \emph{A} to see whether it contains
    a root \emph{y} with the same degree as \emph{x}. If it does, then we link the roots \emph{x} and \emph{y} but
    guaranteeing that \emph{x} remains a root after linking. That is, we link \emph{y} to \emph{x} after first
    exchanging the pointers to the two roots if \emph{y}’s key is smaller than \emph{x}’s key. After
    we link \emph{y} to \emph{x}, the degree of \emph{x} has increased by 1, and so we continue this process,
    linking \emph{x} and another root whose degree equals \emph{x}’s new degree, until no other root
    that we have processed has the same degree as \emph{x}. We then set the appropriate entry
    of \emph{A} to point to \emph{x}, so that as we process roots later on, we have recorded that \emph{x} is
    the unique root of its degree that we have already processed. When this for loop
    terminates, at most one root of each degree will remain, and the array \emph{A} will point
    to each remaining root.

    \begin{algorithm}
        \caption{DeleteMin for Fibonacci heap}
        \LinesNumbered
        \KwIn{Fibonacci heap \emph{H}}
        \emph{z = H.min}\;
        \If{z \textup{!= NULL}}{
            \For{\textup{each child} x \textup{of} z}{
                add \emph{x} to the root list of \emph{H}\;
                \emph{x.p = }NULL
            }
            remove \emph{z} from the root list of \emph{H}\;
        \eIf{z == z.right}{
            \emph{H.min} = NULL\;
        }(\emph{H.min = z.right}){
            consolidate(\emph{H})\;
        }
        \emph{H.n} = \emph{H.n} - 1
        }
    \end{algorithm}

    \begin{algorithm}
        \caption{Consolidate for Fibonacci heap}
        \LinesNumbered
        \KwIn{Fibonacci heap \emph{H}, node \emph{x}}
        let \emph{A} be a new array\;
        \For{i = \textup{0 to size of} A}{
            \emph{A}[\emph{i}] = NULL\;
        }
        \For{\textup{each node} w \textup{in the root list of} H}{
            \emph{x = w}\;
            \emph{d = x.degree}\;
            \While{A\textup{[}d\textup{]} \textup{!= NULL}}{
                \emph{y = A}[\emph{d}]\;
                \If{x.key > y.key}{
                    exchange \emph{x} with \emph{y}\;
                }
                remove \emph{y} from the root list of \emph{H}\;
                make \emph{y} a child of \emph{x}, incrementing \emph{x.degree}\;
                \emph{y.mark} = false\;
                \emph{A}[\emph{d}] = NULL\;
                \emph{d = d} + 1\;
            }
            \emph{A}[\emph{d}] = \emph{x}\;
        }
        \emph{H.min} = NULL\;
        \For{i = \textup{0 to size of} A}{
            \If{A\textup{[}d\textup{] != NULL}}{
                \eIf{H.min == \textup{NULL}}{
                    create a root list for \emph{H} containing just \emph{A}[\emph{i}]\;
                    \emph{H.min} = \emph{A}[\emph{i}]\;
                }{
                    insert \emph{A}[\emph{i}] into \emph{H}'s root list\;
                    \If{A\textup{[}i\textup{]}.key < H.min.key}{
                        \emph{H.min = A}[\emph{i}]\;
                    }
                }
            }
        }
    \end{algorithm}

    \subsubsection{Merge}
    We don't need this procedure in this project, so I just describe
    this procedure without pseudocode. To merge two Fibonacci heaps $H_{1}$ and $H_{2}$, 
    we first concatenate the root lists of $H_{1}$ and $H_{2}$ into a new root list \emph{H}. 
    Then we set the minimum node of \emph{H} and new \emph{H.n}.

    \subsubsection{DecreseKey}
    DecreseKey operation is crucial for the optimization for Dijkstra's algorithm
    because it has only \emph{O}(1) amortized time. To implement the operation, 
    we first judge if the min-heap order has not been violated, if it is, we don't need to adjust the position of the node.
    If min-heap order has been violated, many changes may occur. We start by
    cutting procedure, which “cuts” the link between x and its parent y,
    making x a root.

    We use the mark attributes to obtain the desired time bounds.
    As soon as the second child has been lost, we cut \emph{x} from its parent, making it a new
    root, and sometimes we need cascading-cut operation. Once all the cascading 
    cuts have occurred, the procedure can finish up by updating \emph{H.min} if necessary. The only node whose key changed
    was the node \emph{x} whose key decreased. Thus, the new minimum node is either the
    original minimum node or node \emph{x}.

    \begin{algorithm}
        \caption{DecreseKey for Fibonacci heap}
        \LinesNumbered
        \KwIn{Fibonacci heap \emph{H}, node \emph{x}, new key value \emph{k}}
        \If{k > x.key}{
            \textbf{error}"new key is greater than current key"\;
        }
        \emph{x.key = k}\;
        \emph{y = x.p}\;
        \If{y\textup{!= NULL and }x.key < y.key}{
            remove \emph{x} from the child list of \emph{y}, decrementing \emph{y.degree}\;
            add \emph{x} to the root list of \emph{H}\;
            \emph{x.p} = NULL\;
            \emph{x.mark} = false\;
            cascading\_cut(\emph{H}, \emph{y})\;
        }
        \If{x.key < H.min.key}{
            H.min = x\;
        }
    \end{algorithm}

    \begin{algorithm}
        \caption{Cascade cut for Fibonacci heap}
        \LinesNumbered
        \KwIn{Fibonacci heap \emph{H}, node \emph{y}}
        \emph{z = y.p}\;
        \If{z\textup{!= NULL}}{
            \eIf{y.mark == \textup{false}}{
                \emph{y.mark} = true\;
            }{
                remove \emph{y} from the child list of \emph{z}, decrementing \emph{z.degree}\;
                add \emph{y} to the root list of \emph{H}\;
                \emph{y.p} = NULL\;
                \emph{y.mark} = false\;
                cascading\_cut(\emph{H}, \emph{z})\;
            }
        }
    \end{algorithm}
    \subsubsection{Delete}
    For deletion, we just need to use decrease key function to decrease
    the data stored in the node we want to delete to a small value that 
    normal node won't store. 

    \section{Algorithm Analysis}
    \subsection{Heap Operations}
    In this section, we define n as the number of elements in 
    priority queue. And the result in the table represent the 
    worst time, \textbf{except the column for Fibonacci heap, which
    represent amortized time}. 
    \begin{table}[h]
        \centering
		\begin{tabular}{l l l l l}
			
			\textbf{Operation} & \textbf{Binary heap} & \textbf{Leftist heap} & \textbf{Binomial heap} & \textbf{Fibonacci heap}\\
			
			Make heap    & \emph{O}(\emph{n}) & \emph{O}(\emph{n}) & \emph{O}(\emph{n}) & \emph{O}(\emph{n})\\
			Find min     & \emph{O}(1)        & \emph{O}(1)        & \emph{O}(1)        & \emph{O}(1)\\
			Insert       & \emph{O}(log\emph{n}) & \emph{O}(log\emph{n}) & \emph{O}(log\emph{n}) & \emph{O}(1)\\
            Delete min   & \emph{O}(log\emph{n}) & \emph{O}(log\emph{n}) & \emph{O}(log\emph{n}) & \emph{O}(log\emph{n})\\
            Merge        & \emph{O}(\emph{n}) & \emph{O}(log\emph{n}) & \emph{O}(log\emph{n}) & \emph{O}(1)\\
            Delete       & \emph{O}(log\emph{n}) & \emph{O}(log\emph{n}) & \emph{O}(log\emph{n}) & \emph{O}(log\emph{n})\\
            Decrese key  & \emph{O}(log\emph{n}) & \emph{O}(log\emph{n}) & \emph{O}(log\emph{n}) & \emph{O}(1)
		\end{tabular}
		\caption{Running times of all operations of different heaps}
	\end{table}
    \subsection{optimization for Dijkstra's Algorithm}
    We have learned that theoretically, the total running time of 
    Dijkstra's algorithm without priority queue is $O(|E|+|V|^{2})$. 
    So if the graph is dense, with $|E| = \Theta(|V|^{2})$, the algorithm 
    without priority queue is good enough.

    But in practice, also in this project, the grapg is always sparse, 
    which means $|E| = \Theta(|V|)$, the algorithm is too slow. 
    So we consider the heap operation DeleteMin, whose running time is 
    $O(n\textup{log}n)$ in all the heaps we use in the project, to find and delete the
    minimum vertex. And we use DecreseKey, or in practice insertion, whose 
    running time is $O(n\textup{log}n)$ for binary heap, leftist heap,  
    binomial heap, and $O(1)$ for Fibonacci heap.

    The improtant observation is that the running time of Dijkstra algorithm 
    is dominated by $|E|$ DecreseKey/Insertion operation and $|V|$ DeleteMin 
    operations. So theoretically, for binary heap, leftist heap and binomial 
    heap, the time bound for the optimized Dijkstra algorithm is 
    $O(|E|\textup{log}|V| + |V|\textup{log}|V|) = O(|E|\textup{log}|V|)$, and for Fibonacci heap, 
    the time bound will be decreased to $O(|E| + |V|\textup{log}|V|)$.

    \section{Experiment and Result}
        
    \section{Discussion and Conclusion}
        
\end{document}